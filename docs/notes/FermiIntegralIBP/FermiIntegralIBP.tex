\documentclass[11pt,a4paper]{article}

\usepackage[utf8]{inputenc}
\usepackage{amsmath,amssymb,amsfonts}
\usepackage{bm}
\usepackage{physics}
\usepackage{booktabs}
\usepackage{array}
\usepackage{geometry}
\geometry{margin=2.5cm}
\usepackage{hyperref}
\usepackage{cleveref}

\newcommand{\pF}{p_{\rm F}}
\newcommand{\GeV}{\,\text{GeV}}
\newcommand{\fm}{\,\text{fm}}
\newcommand{\MeV}{\,\text{MeV}}
\newcommand{\vep}{\varepsilon}

\title{Integration-by-parts improvement of Fermi integrals \\
for baryon number susceptibilities at low temperature}
\author{Technical note for Thermal-FIST}
\date{\today}

\begin{document}
\maketitle

\begin{abstract}
We describe the integration-by-parts (IBP) technique used to improve the numerical evaluation of Fermi-Dirac integrals for baryon number susceptibilities $\chi_n^B$ at low temperatures in the hadron resonance gas model.
The key idea is to reduce the order of the Fermi--Dirac distribution derivatives appearing in the integrand by one, replacing oscillatory or sharply peaked integrands with smoother ones that are better resolved by the existing Sommerfeld--Legendre + Laguerre quadrature scheme.
We present the IBP formulas for $\chi_2$, $\chi_3$, and $\chi_4$, derive the analytic $T=0$ limits, and document the numerical improvements achieved.
\end{abstract}

\tableofcontents

%% ====================================================================
\section{Setup and notation}
\label{sec:setup}
%% ====================================================================

Consider a single fermion species with mass~$m$, degeneracy~$g$, at temperature~$T$ and chemical potential~$\mu > m$ (so that the Fermi momentum $\pF = \sqrt{\mu^2 - m^2}$ is real).
The Fermi--Dirac distribution is
\begin{equation}
  f(p) = \frac{1}{e^{(E-\mu)/T} + 1}\,, \qquad E = \sqrt{p^2 + m^2}\,.
\end{equation}
The particle number density is
\begin{equation}
  n = \frac{g}{2\pi^2} \int_0^\infty p^2 \, f(p) \, dp\,.
\end{equation}
The generalized susceptibilities are the $\mu$-derivatives of the pressure, or equivalently:
\begin{equation}
  \chi_n \equiv \frac{\partial^n P}{\partial \mu^n} = \frac{\partial^{n-1} n}{\partial \mu^{n-1}}\,.
\end{equation}
Using the identity $\partial f / \partial \mu = f(1-f)/T$, the first few susceptibilities involve
\begin{align}
  \chi_2 &= \pdv{n}{\mu} = \frac{g}{2\pi^2} \int_0^\infty p^2 \, \frac{f(1-f)}{T} \, dp\,, \label{eq:chi2-direct} \\[6pt]
  \chi_3 &= \pdv[2]{n}{\mu} = \frac{g}{2\pi^2} \int_0^\infty p^2 \, \frac{f(1-f)(1-2f)}{T^2} \, dp\,, \label{eq:chi3-direct} \\[6pt]
  \chi_4 &= \pdv[3]{n}{\mu} = \frac{g}{2\pi^2} \int_0^\infty p^2 \, \frac{f(1-f)\bigl[1 - 6f(1-f)\bigr]}{T^3} \, dp\,. \label{eq:chi4-direct}
\end{align}

\paragraph{The numerical problem.}
As $T \to 0$, the factor $f(1-f)/T$ tends to $\delta(E-\mu)$, i.e.\ $\delta(p-\pF)/v_F$ with $v_F = \pF/\mu$.
For $\chi_2$ the integrand is a narrow positive peak at~$\pF$; for $\chi_3$ the integrand $f(1-f)(1-2f)/T^2$ has a derivative-of-delta structure (one sign change); for $\chi_4$ the integrand $f(1-f)[1-6f(1-f)]/T^3$ has a second-derivative-of-delta structure (two sign changes).
Higher-order integrands are increasingly oscillatory and harder for fixed-order quadrature to resolve.

We use a hybrid quadrature scheme:
\begin{itemize}
  \item \textbf{Sommerfeld--Legendre} (32 points): maps $[0,1] \to [0, \pF]$ with adaptive concentration of quadrature nodes near the Fermi surface, controlled by $\alpha = \pF^2/(\mu T)$.
  \item \textbf{Shifted Laguerre} (32 points): covers $[\pF, \infty)$ with exponential weight.
\end{itemize}
Both sets of nodes are concentrated near~$\pF$, where the Fermi--Dirac integrands peak.
However, for the oscillatory integrands in $\chi_3$ and especially $\chi_4$, the quadrature accuracy degrades severely at low~$T$.
The IBP technique described below smooths the integrands by reducing the derivative order by one.


%% ====================================================================
\section{IBP identities}
\label{sec:ibp}
%% ====================================================================

The core identity underlying all IBPs is
\begin{equation}\label{eq:dfdt-identity}
  \frac{df}{dp} = -\frac{p}{ET}\,f(1-f)\,.
\end{equation}
This allows us to express products of $f$ and its $\mu$-derivatives as total $p$-derivatives, enabling integration by parts.

\subsection{\texorpdfstring{$\chi_2$}{chi2}: first IBP}
\label{sec:chi2}

From \cref{eq:dfdt-identity}, we have
\begin{equation}
  f(1-f) = -T\,\frac{E}{p}\,\frac{df}{dp}\,.
\end{equation}
Therefore the $\chi_2$ integrand becomes a total derivative:
\begin{equation}
  p^2 \, f(1-f) = -T\,p\,E\,\frac{df}{dp} = -T\,\frac{d}{dp}\bigl[p\,E\,f\bigr] + T\,\frac{d(pE)}{dp}\,f\,.
\end{equation}
Since $d(pE)/dp = (2p^2 + m^2)/E$ and the boundary term $[p\,E\,f]_0^\infty = 0$ (vanishes at $p=0$ and exponentially at $p=\infty$), integration by parts gives
\begin{equation}\label{eq:chi2-ibp}
  \boxed{\int_0^\infty p^2\,f(1-f)\,dp = T \int_0^\infty \frac{2p^2 + m^2}{E}\,f\,dp\,.}
\end{equation}
The left-hand side has integrand $f(1{-}f) \sim \delta(p{-}\pF)$; the right-hand side has integrand $f$, the Fermi step function itself --- smooth, monotonic, and trivial for quadrature.

Using \cref{eq:chi2-ibp} in \cref{eq:chi2-direct}:
\begin{equation}\label{eq:chi2-final}
  \chi_2 = \frac{g}{2\pi^2} \int_0^\infty \frac{2p^2 + m^2}{E}\,f\,dp\,.
\end{equation}


\subsection{\texorpdfstring{$\chi_3$}{chi3}: second IBP}
\label{sec:chi3}

Differentiating $f(1-f) = -T(E/p)\,df/dp$ once more with respect to~$\mu$:
\begin{equation}
  \frac{\partial}{\partial\mu}\bigl[f(1-f)\bigr] = \frac{f(1-f)(1-2f)}{T} \quad \Rightarrow \quad f(1-f)(1-2f) = -T\,\frac{E}{p}\,\frac{d}{dp}\bigl[f(1-f)\bigr]\,.
\end{equation}
Therefore
\begin{equation}
  p^2\,f(1-f)(1-2f) = -T\,p\,E\,\frac{d}{dp}\bigl[f(1-f)\bigr]\,.
\end{equation}
Integrating by parts with $[p\,E\,f(1{-}f)]_0^\infty = 0$ (the boundary terms vanish since $pE = 0$ at $p=0$ and $f(1-f)$ decays exponentially at $p \to \infty$):
\begin{equation}\label{eq:chi3-ibp}
  \boxed{\int_0^\infty p^2\,f(1-f)(1-2f)\,dp = T \int_0^\infty \frac{2p^2 + m^2}{E}\,f(1-f)\,dp\,.}
\end{equation}
The left-hand side has an integrand with one sign change (derivative-of-delta structure); the right-hand side has $f(1{-}f)$, a smooth non-negative peak at~$\pF$ --- the same integrand that appeared in the \emph{un-improved} $\chi_2$.

Using \cref{eq:chi3-ibp} in \cref{eq:chi3-direct}:
\begin{equation}\label{eq:chi3-final}
  \chi_3 = \frac{g}{2\pi^2 T} \int_0^\infty \frac{2p^2 + m^2}{E}\,f(1-f)\,dp\,.
\end{equation}


\subsection{\texorpdfstring{$\chi_4$}{chi4}: third IBP}
\label{sec:chi4}

Similarly, differentiating $f(1-f)(1-2f)$ with respect to~$\mu$:
\begin{equation}
  \frac{\partial}{\partial\mu}\bigl[f(1-f)(1-2f)\bigr] = \frac{f(1-f)\bigl[1-6f(1-f)\bigr]}{T}
\end{equation}
and thus
\begin{equation}
  f(1-f)\bigl[1 - 6f(1-f)\bigr] = -T\,\frac{E}{p}\,\frac{d}{dp}\bigl[f(1-f)(1-2f)\bigr]\,.
\end{equation}
Integration by parts with $[p\,E\,f(1{-}f)(1{-}2f)]_0^\infty = 0$ gives:
\begin{equation}\label{eq:chi4-ibp}
  \boxed{\int_0^\infty p^2\,f(1-f)\bigl[1-6f(1-f)\bigr]\,dp = T \int_0^\infty \frac{2p^2 + m^2}{E}\,f(1-f)(1-2f)\,dp\,.}
\end{equation}
The left-hand side has two sign changes; the right-hand side has one sign change (the same integrand as the \emph{un-improved} $\chi_3$).

Using \cref{eq:chi4-ibp} in \cref{eq:chi4-direct}:
\begin{equation}\label{eq:chi4-final}
  \chi_4 = \frac{g}{2\pi^2 T^2} \int_0^\infty \frac{2p^2 + m^2}{E}\,f(1-f)(1-2f)\,dp\,.
\end{equation}

\subsection{General pattern}
\label{sec:general}

The IBP has a recursive structure.
Define
\begin{equation}
  F_k(p) \equiv T \frac{\partial^k f}{\partial \mu^k}\,,
\end{equation}
so that $F_0 = f$, $F_1 = f(1-f)$, $F_2 = f(1-f)(1-2f)/T$, etc.
Then the identity $\partial f/\partial p = -(p/ET)\,F_1$ generalizes to
\begin{equation}
  F_{k+1} = \frac{\partial F_k}{\partial\mu} = -T\,\frac{E}{p}\,\frac{\partial F_k}{\partial p}\,,
\end{equation}
and each IBP step replaces
\begin{equation}
  \int_0^\infty p^2 \, F_{k+1}\, dp = T \int_0^\infty \frac{2p^2+m^2}{E}\,F_k\,dp\,,
\end{equation}
provided the boundary term $[pE\,F_k]_0^\infty$ vanishes.
Each step reduces the number of sign changes in the integrand by one, at the cost of introducing one additional power of~$T$ in the denominator when forming~$\chi_n$.


%% ====================================================================
\section{Analytic \texorpdfstring{$T=0$}{T=0} limits}
\label{sec:T0}
%% ====================================================================

At $T = 0$, the Fermi--Dirac distribution becomes a step function, $f = \theta(\pF - p)$, and the integrals can be evaluated analytically.

\subsection{Density and equation of state}

\begin{align}
  n\big|_{T=0} &= \frac{g}{2\pi^2}\,\frac{\pF^3}{3}\,, \\[4pt]
  P\big|_{T=0} &= \frac{g}{6\pi^2}\left[\mu\,\pF^3 - \tfrac{3}{4}\,\pF^4\,\psi(m/\pF)\right], \\[4pt]
  \vep\big|_{T=0} &= \frac{g}{2\pi^2}\,\frac{\pF^4}{4}\,\psi(m/\pF)\,,
\end{align}
where $\psi(x) = (1+x^2)^{1/2}(1 + \tfrac{1}{2}x^2) - \tfrac{1}{2}x^4\sinh^{-1}(1/x)$.

\subsection{Susceptibilities}

Using $f(1{-}f)/T \to \delta(p - \pF)\,\mu/\pF$ and its derivatives, or by directly differentiating the $T=0$ density:
\begin{align}
  \chi_2\big|_{T=0} = \frac{dn}{d\mu}\bigg|_{T=0} &= \frac{g}{2\pi^2}\,\frac{\mu\,\pF}{1} = \frac{g}{2\pi^2}\,\mu\,\pF\,, \label{eq:chi2-T0} \\[6pt]
  \chi_3\big|_{T=0} = \frac{d^2 n}{d\mu^2}\bigg|_{T=0} &= \frac{g}{2\pi^2}\,\frac{\mu^2 + \pF^2}{\pF}\,, \label{eq:chi3-T0} \\[6pt]
  \chi_4\big|_{T=0} = \frac{d^3 n}{d\mu^3}\bigg|_{T=0} &= \frac{g}{2\pi^2}\,\frac{\mu\,(3\pF^2 - \mu^2)}{\pF^3}\,. \label{eq:chi4-T0}
\end{align}

These analytic values serve as reference for assessing the numerical accuracy at finite but small~$T$.


%% ====================================================================
\section{Decomposition into below- and above-\texorpdfstring{$\pF$}{pF} contributions}
\label{sec:decomposition}
%% ====================================================================

The quadrature scheme splits every integral at the Fermi momentum~$\pF$:
\begin{equation}\label{eq:split}
  \int_0^\infty (\ldots)\,dp = \underbrace{\int_0^{\pF} (\ldots)\,dp}_{\text{below }\pF} \;+\; \underbrace{\int_{\pF}^\infty (\ldots)\,dp}_{\text{above }\pF}\,.
\end{equation}
Since $f \to \theta(\pF - p)$ as $T \to 0$, the above-$\pF$ part vanishes exponentially, while the below-$\pF$ part approaches a $T$-independent limit.
We now show how this split interacts with the IBP integrands of each susceptibility.

\subsection{\texorpdfstring{$\chi_2$}{chi2}: the \texorpdfstring{$T=0$}{T=0} value emerges explicitly}
\label{sec:chi2-decomp}

The IBP formula for $\chi_2$ (\cref{eq:chi2-ibp}) gives
\begin{equation}
  T\,\chi_2 = \frac{g}{2\pi^2} \int_0^\infty \frac{2p^2 + m^2}{E}\,f\,dp\,.
\end{equation}
The integrand involves $f$ --- the Fermi step function.
Below~$\pF$ we write $f = 1 - (1-f)$, obtaining
\begin{equation}\label{eq:chi2-below}
  \int_0^{\pF} \frac{2p^2 + m^2}{E}\,f\,dp
  = \underbrace{\int_0^{\pF} \frac{2p^2 + m^2}{E}\,dp}_{= \;\mu\,\pF}
  \;-\; \int_0^{\pF} \frac{2p^2 + m^2}{E}\,(1-f)\,dp\,.
\end{equation}
The first integral is the \emph{exact} $T = 0$ contribution and can be evaluated analytically:
\begin{equation}\label{eq:chi2-analytic}
  \int_0^{\pF} \frac{2p^2 + m^2}{E}\,dp
  = \int_0^{\pF} \left(2E - \frac{m^2}{E}\right) dp
  = \bigl[p\,E\bigr]_0^{\pF} = \pF \cdot \mu = \mu\,\pF\,,
\end{equation}
where we used $d(pE)/dp = (2p^2 + m^2)/E$ and the boundary values $p\,E|_{p=\pF} = \pF\,\mu$, $p\,E|_{p=0} = 0$.

The remaining terms are thermal corrections that vanish as $T \to 0$:
\begin{itemize}
  \item $-\int_0^{\pF} \frac{2p^2+m^2}{E}\,(1-f)\,dp$: the ``hole'' contribution below~$\pF$.
  At $T = 0$, $1 - f = 0$ for $p < \pF$, so this integral vanishes.
  At finite $T$, $1 - f$ is exponentially small except near $\pF$, contributing $O(T^2)$.
  \item $+\int_{\pF}^\infty \frac{2p^2+m^2}{E}\,f\,dp$: the ``tail'' contribution above~$\pF$.
  Again $f \to 0$ for $p > \pF$ as $T \to 0$, giving $O(T^2)$.
\end{itemize}
The code therefore computes
\begin{equation}\label{eq:chi2-code}
  T\,\chi_2 = \frac{g}{2\pi^2}\,T\,\bigl(\underbrace{\mu\,\pF}_{\text{ret2}} + \underbrace{\Delta_{\rm th}}_{\text{ret1}}\bigr)\,,
\end{equation}
where $\text{ret2} = \mu\,\pF$ is the analytic $T = 0$ value of $\chi_2$ (in natural units, up to the $g/2\pi^2$ prefactor) and
\begin{equation}
  \Delta_{\rm th} = -\int_0^{\pF} \frac{2p^2+m^2}{E}\,(1-f)\,dp + \int_{\pF}^\infty \frac{2p^2+m^2}{E}\,f\,dp
\end{equation}
is the thermal correction, which tends to zero as $T \to 0$.

\paragraph{Key point.}
The analytic extraction for $\chi_2$ is \emph{not} a numerical convenience --- it is a structural consequence of the IBP.
The IBP replaces the sharply peaked integrand $f(1-f)$ by the step-function-like $f$, which \emph{naturally} splits into a filled Fermi sea (yielding the exact $T = 0$ answer) plus corrections from particles and holes near the Fermi surface.

\subsection{\texorpdfstring{$\chi_3$}{chi3}: peaked integrand, no natural \texorpdfstring{$T=0$}{T=0} split}
\label{sec:chi3-decomp}

The IBP formula for $\chi_3$ (\cref{eq:chi3-ibp}) gives
\begin{equation}
  T^2\,\chi_3 = \frac{g}{2\pi^2}\,T \int_0^\infty \frac{2p^2 + m^2}{E}\,f(1-f)\,dp\,.
\end{equation}
The integrand $f(1{-}f)$ is a smooth, non-negative peak localized within $\sim T$ of $\pF$.
Unlike the $\chi_2$ case, $f(1{-}f)$ vanishes at $T = 0$ for all~$p$ (both below and above $\pF$), so there is no filled-Fermi-sea piece to extract directly from the integrand.

Instead, the integral scales as $\sim T$ for small $T$:
\begin{equation}
  \int_0^\infty \frac{2p^2 + m^2}{E}\,f(1-f)\,dp \;\xrightarrow{T \to 0}\; T\,\frac{\mu^2 + \pF^2}{\pF}\,,
\end{equation}
because $f(1{-}f)/T \to \delta(p - \pF)\,\mu/\pF$.
Therefore $\chi_3 = \text{quad}/T \to (\mu^2 + \pF^2)/\pF$ as $T \to 0$.

For the $T = 0$ extraction, the code computes
\begin{equation}\label{eq:chi3-code}
  T^2\,\chi_3 = \frac{g}{2\pi^2}\,T^2 \bigl(\underbrace{(\mu^2 + \pF^2)/\pF}_{\text{ret2}} + \underbrace{\text{quad}/T - (\mu^2 + \pF^2)/\pF}_{\text{ret1}}\bigr)\,.
\end{equation}
Here $\text{ret2}$ is the known $T = 0$ limit and $\text{ret1}$ is the thermal correction.
Unlike the $\chi_2$ case, $\text{ret2}$ does not emerge organically from splitting the integral at $\pF$ --- it is an independently computed constant that is subtracted.
This decomposition is mathematically equivalent to the direct computation ($T^2 \times \text{value}/T^2$ round-trips losslessly in IEEE double), but it makes the $T \to 0$ behavior transparent.

\subsection{\texorpdfstring{$\chi_4$}{chi4}: sign-changing integrand}
\label{sec:chi4-decomp}

The IBP formula for $\chi_4$ (\cref{eq:chi4-ibp}) gives
\begin{equation}
  T^3\,\chi_4 = \frac{g}{2\pi^2}\,T \int_0^\infty \frac{2p^2 + m^2}{E}\,f(1-f)(1-2f)\,dp\,.
\end{equation}
The integrand $f(1{-}f)(1{-}2f)$ is localized near $\pF$ and changes sign once (positive for $p < \pF$, negative for $p > \pF$, since $1 - 2f$ flips sign at $E = \mu$).
Like $\chi_3$, the integrand vanishes everywhere at $T = 0$, and the integral scales as $\sim T^2$:
\begin{equation}
  \int_0^\infty \frac{2p^2 + m^2}{E}\,f(1-f)(1-2f)\,dp \;\xrightarrow{T \to 0}\; T^2\,\frac{\mu(3\pF^2 - \mu^2)}{\pF^3}\,.
\end{equation}
The code uses the same extraction structure:
\begin{equation}\label{eq:chi4-code}
  T^3\,\chi_4 = \frac{g}{2\pi^2}\,T^3 \bigl(\underbrace{\mu(3\pF^2 - \mu^2)/\pF^3}_{\text{ret2}} + \underbrace{\text{quad}/T^2 - \mu(3\pF^2 - \mu^2)/\pF^3}_{\text{ret1}}\bigr)\,.
\end{equation}
Because $f(1{-}f)(1{-}2f)$ has one sign change, the quadrature accuracy is intermediate between $\chi_2$ (no sign changes, excellent) and the un-improved $\chi_4$ (two sign changes, catastrophic).

\subsection{Comparison of the three cases}

The qualitative difference between $\chi_2$ and $\chi_{3,4}$ is summarized in the following table:
\begin{center}
\renewcommand{\arraystretch}{1.4}
\begin{tabular}{@{}lccc@{}}
\toprule
 & $\chi_2$ & $\chi_3$ & $\chi_4$ \\
\midrule
IBP integrand & $f$ & $f(1{-}f)$ & $f(1{-}f)(1{-}2f)$ \\
Sign changes & 0 & 0 & 1 \\
$T\to 0$ behavior of integrand & $\to \theta(\pF - p)$ & $\to 0$ (peak $\sim T$) & $\to 0$ (peak $\sim T$) \\
$T{=}0$ from integral splitting? & \textbf{Yes} & No & No \\
How $T{=}0$ enters & $\int_0^{\pF} \frac{2p^2+m^2}{E}\,dp = \mu\pF$ & $\text{quad}/T \to \text{const}$ & $\text{quad}/T^2 \to \text{const}$ \\
\bottomrule
\end{tabular}
\end{center}
For $\chi_2$, the IBP is special: the integrand $f$ has a non-vanishing $T = 0$ limit (the filled Fermi sea), and splitting $f = 1 - (1-f)$ below~$\pF$ yields the analytic answer \emph{organically}.
For $\chi_3$ and $\chi_4$, the integrands are purely thermal ($\to 0$ everywhere as $T \to 0$), so the $T = 0$ values emerge only through the scaling of the integral with powers of~$T$, and the analytic extraction is a separate subtraction rather than a structural decomposition.


%% ====================================================================
\section{Implementation details}
\label{sec:implementation}
%% ====================================================================

\subsection{Quadrature scheme}

The integration domain $[0,\infty)$ is split at the Fermi momentum $\pF$:

\paragraph{Below $\pF$: Sommerfeld--Legendre quadrature.}
A nonlinear mapping $s \in [0,1] \to p \in [0, \pF]$ concentrates quadrature nodes near the Fermi surface:
\begin{equation}
  p(s) = \pF\left(1 - \frac{u(s)}{\alpha}\right), \qquad
  u(s) = -\ln\bigl(1 - \beta\,(1-s)\bigr), \qquad
  \alpha = \frac{\pF^2}{\mu T}, \qquad \beta = 1 - e^{-\alpha}\,.
\end{equation}
This is applied with 32-point Gauss--Legendre nodes and weights on~$[0,1]$.
For $\alpha \ll 1$ (i.e.\ $T \gg \pF^2/\mu$, or equivalently $T \gg$ Fermi energy), the mapping reduces to uniform: $p = s\,\pF$.

\paragraph{Above $\pF$: shifted Laguerre quadrature.}
The substitution $p = \pF + T\,t$ transforms the Laguerre integration variable $t \in [0,\infty)$ with 32-point Gauss--Laguerre nodes and weights.
The Jacobian $dp = T\,dt$ contributes a factor~$T$.

\subsection{Below-\texorpdfstring{$\pF$}{pF} forms}

For $p < \pF$, we have $E < \mu$ so $x = (E-\mu)/T < 0$.
The Fermi--Dirac products are evaluated using $e^x$ (which is $< 1$):
\begin{align}
  f &= \frac{1}{e^x + 1}\,, \\[4pt]
  f(1{-}f) &= \frac{e^x}{(1+e^x)^2}\,, \\[4pt]
  f(1{-}f)(1{-}2f) &= \frac{e^x}{(1+e^x)^2}\left(1 - \frac{2}{e^x+1}\right) = \frac{e^x(e^x - 1)}{(1+e^x)^3}\,.
\end{align}
These forms are numerically stable for all $x < 0$.

For $\chi_2$, the code computes
\begin{equation}
  -\int_0^{\pF} \frac{2p^2+m^2}{E}\,\frac{1}{e^{-(E-\mu)/T}+1}\,dp\,,
\end{equation}
where the negative sign and $e^{-x}$ in the denominator implement the ``hole'' integrand $-(1-f) = -(e^{-x}/(1+e^{-x})) = -1/(e^{-(E-\mu)/T}+1)$.
This equals $-\int_0^{\pF} (2p^2+m^2)/E \cdot (1-f)\,dp$, which when combined with the analytic $\mu\,\pF$ from $\int_0^{\pF} (2p^2+m^2)/E\,dp$ gives the total below-$\pF$ contribution.

\subsection{Above-\texorpdfstring{$\pF$}{pF} forms}

For $p > \pF$, we have $x = (E-\mu)/T > 0$.
The substitution $p = \pF + T\,t$ maps the integration to Laguerre variables $t \in [0,\infty)$, and the Fermi--Dirac products are evaluated using $e^{-x}$ (which is $< 1$):
\begin{align}
  f &= \frac{e^{-x}}{1+e^{-x}}\,, \\[4pt]
  f(1{-}f) &= \frac{e^{-x}}{(1+e^{-x})^2}\,, \\[4pt]
  f(1{-}f)(1{-}2f) &= \frac{e^{-x}(1 - e^{-x})}{(1+e^{-x})^3}\,.
\end{align}

In terms of the scaled variable $t$, the momentum is $p = T\,t$, the energy-over-$T$ is $E/T = \sqrt{t^2 + (m/T)^2}$, and $x = E/T - \mu/T$.
The kinematic factor becomes
\begin{equation}
  \frac{2p^2+m^2}{E}\,dp = T^2\,\frac{2t^2 + (m/T)^2}{E/T}\,dt\,.
\end{equation}
The extra factor of $T^2$ (one from $dp = T\,dt$, one from $2p^2/E = 2T^2 t^2/(ET)$, but the $m^2/E$ term contributes less) appears in the code as the explicit \texttt{T*T} multiplying the Laguerre contribution.

\subsection{Analytic \texorpdfstring{$T=0$}{T=0} extraction}

The code stores each susceptibility in the form
\begin{equation}
  T^k \, \chi_{k+1} = \frac{g}{2\pi^2}\,T^k\bigl(\underbrace{\chi_{k+1}^{(0)}}_{\text{ret2}} + \underbrace{\Delta\chi_{k+1}}_{\text{ret1}}\bigr)\,,
\end{equation}
where $\chi_{k+1}^{(0)}$ is the analytic $T=0$ value from \cref{sec:T0} and $\Delta\chi_{k+1} = \text{quad}/T^k - \chi_{k+1}^{(0)}$ is the thermal correction.

\begin{itemize}
\item For \textbf{$\chi_2$} ($k = 1$): the extraction is structural.
  The below-$\pF$ quadrature computes the hole contribution
  $-\int_0^{\pF}(2p^2+m^2)/E\cdot(1-f)\,dp$,
  the above-$\pF$ quadrature computes the tail
  $+\int_{\pF}^\infty(2p^2+m^2)/E\cdot f\,dp$,
  and $\text{ret2} = \mu\,\pF$ is the exact Fermi-sea integral.
  The sum $\text{ret1} + \text{ret2}$ equals the full $\chi_2$ value to machine precision.

\item For \textbf{$\chi_3$} ($k = 2$): both quadratures compute $\int (2p^2+m^2)/E \cdot f(1{-}f)\,dp$ directly (no $f = 1 - (1-f)$ decomposition, since $f(1{-}f) \to 0$ below~$\pF$).
  The analytic value $\text{ret2} = (\mu^2 + \pF^2)/\pF$ is subtracted after dividing the total quadrature by~$T$.

\item For \textbf{$\chi_4$} ($k = 3$): similarly, both quadratures compute $\int (2p^2+m^2)/E \cdot f(1{-}f)(1{-}2f)\,dp$ directly, and $\text{ret2} = \mu(3\pF^2 - \mu^2)/\pF^3$ is subtracted after dividing by $T^2$.
\end{itemize}

Since $T^k \times \text{value} / T^k$ round-trips without precision loss in IEEE-754 double arithmetic (no overflow or underflow for $T \gtrsim 10^{-6}\GeV$), this decomposition is numerically equivalent to the direct computation in all cases.

\subsection{Summary of implemented formulas}

The code computes $T^k\,d^k n/d\mu^k$ for $k = 1,2,3$ using the IBP-improved integrands:
\begin{center}
\renewcommand{\arraystretch}{1.5}
\begin{tabular}{@{}lccc@{}}
\toprule
\textbf{Quantity} & \textbf{Code returns} & \textbf{IBP integrand} & \textbf{$T{=}0$ value of $d^kn/d\mu^k$} \\
\midrule
$\chi_2 = dn/d\mu$ & $T\,dn/d\mu$ & $(2p^2{+}m^2)/E \cdot f$ & $\mu\,\pF$ \\
$\chi_3 = d^2n/d\mu^2$ & $T^2\,d^2n/d\mu^2$ & $(2p^2{+}m^2)/E \cdot f(1{-}f)$ & $(\mu^2{+}\pF^2)/\pF$ \\
$\chi_4 = d^3n/d\mu^3$ & $T^3\,d^3n/d\mu^3$ & $(2p^2{+}m^2)/E \cdot f(1{-}f)(1{-}2f)$ & $\mu(3\pF^2{-}\mu^2)/\pF^3$ \\
\bottomrule
\end{tabular}
\end{center}
All integrands include the common prefactor $g/(2\pi^2)$ and an overall factor of~$T$ from the IBP (absorbed into the $T^k$ prefactor).
The density ($n$, $k=0$), pressure ($P$), and energy density ($\vep$) do not use IBP --- their integrands ($p^2 f$, $p^4 f/E$, $p^2 E f$) are already smooth.


%% ====================================================================
\section{Numerical results}
\label{sec:results}
%% ====================================================================

We test the IBP improvements using the \texttt{ZeroTemperatureComparison} example at $\mu_B = 1.2\GeV$ with the PDG2025 particle list, comparing the Ideal HRG, excluded-volume diagonal (EV), quantum van~der~Waals (QvdW), and real gas (CS+VDW) models.
All models use quantum statistics with 32+32 quadrature points.

\Cref{tab:chi3,tab:chi4} show relative deviations of the finite-$T$ results from the exact $T=0$ values.

\begin{table}[h]
\centering
\caption{Relative deviation $(\chi_3(T) - \chi_3(0))/\chi_3(0)$ at various temperatures.}
\label{tab:chi3}
\renewcommand{\arraystretch}{1.2}
\begin{tabular}{@{}l cc cc@{}}
\toprule
& \multicolumn{2}{c}{\textbf{Old (Laguerre only)}} & \multicolumn{2}{c}{\textbf{New (IBP)}} \\
\cmidrule(lr){2-3}\cmidrule(lr){4-5}
\textbf{Model} & $T{=}0.001\MeV$ & $T{=}1\MeV$ & $T{=}0.001\MeV$ & $T{=}1\MeV$ \\
\midrule
Ideal HRG        & $-1.000$     & $-1.000$     & $-1.4\times 10^{-4}$  & $4.3\times 10^{-2}$ \\
EV-Diagonal      & $0.294$      & $0.294$      & $3.5\times 10^{-9}$   & $-1.9\times 10^{-5}$ \\
QvdW             & $0.202$      & $0.202$      & $2.9\times 10^{-9}$   & $2.1\times 10^{-4}$ \\
RealGas (CS+VDW) & $0.879$      & $0.879$      & $2.4\times 10^{-8}$   & $-3.6\times 10^{-4}$ \\
\bottomrule
\end{tabular}
\end{table}

\begin{table}[h]
\centering
\caption{Relative deviation $(\chi_4(T) - \chi_4(0))/\chi_4(0)$ at various temperatures.}
\label{tab:chi4}
\renewcommand{\arraystretch}{1.2}
\begin{tabular}{@{}l cc cc@{}}
\toprule
& \multicolumn{2}{c}{\textbf{Old (Laguerre only)}} & \multicolumn{2}{c}{\textbf{New (IBP)}} \\
\cmidrule(lr){2-3}\cmidrule(lr){4-5}
\textbf{Model} & $T{=}0.001\MeV$ & $T{=}1\MeV$ & $T{=}0.001\MeV$ & $T{=}1\MeV$ \\
\midrule
Ideal HRG        & $-1.000$     & $-1.000$     & $1.46$                 & $-8.7\times 10^{-2}$ \\
EV-Diagonal      & $0.846$      & $0.846$      & $1.6\times 10^{-4}$    & $9.7\times 10^{-5}$ \\
QvdW             & $0.505$      & $0.505$      & $6.8\times 10^{-5}$    & $3.3\times 10^{-4}$ \\
RealGas (CS+VDW) & $3.905$      & $3.904$      & $1.7\times 10^{-3}$    & $4.2\times 10^{-4}$ \\
\bottomrule
\end{tabular}
\end{table}

\paragraph{Key observations.}

\begin{enumerate}
  \item \textbf{$\chi_3$ (one IBP step):} The old code produced 20--100\% errors for all models below $T \sim 1\MeV$, with the Ideal HRG returning essentially zero (the Laguerre-only quadrature completely misses the Fermi surface peak).
  After IBP, the integrand $f(1{-}f)$ is a smooth, non-negative peak, and the errors drop to $10^{-9}$--$10^{-4}$ at $T = 0.001\MeV$.

  \item \textbf{$\chi_4$ (one IBP step):} The old code had 50--390\% errors.
  After IBP, the integrand $f(1{-}f)(1{-}2f)$ still has one sign change, so the improvement is less dramatic but still substantial: errors drop to $10^{-5}$--$10^{-3}$ for interacting models.
  The Ideal HRG at $T = 0.001\MeV$ remains challenging (146\% error) because many species with large Fermi momenta contribute, and the sign-changing integrand is harder to resolve per species.
  By $T = 0.1\MeV$ the Ideal HRG error is already below~$0.4\%$.

  \item \textbf{The old errors are flat in $T$:} For the interacting models with the old code, the relative deviations are essentially constant from $T = 0.001\MeV$ to $T = 0.1\MeV$.
  This confirms the errors are systematic quadrature failures (the Laguerre nodes do not resolve the Fermi surface), not physical effects.

  \item \textbf{A second IBP for $\chi_4$ is not possible.} One might attempt to apply the IBP once more to replace $f(1{-}f)(1{-}2f)$ with $f(1{-}f)$.
  However, the boundary term $[pE \cdot f(1{-}f)]_0^\infty$ does vanish, but the resulting integrand involves $d[(2p^2{+}m^2)/E]/dp = p(p^2 + 2m^2)/E^3$, which when combined with $pE \cdot f(1{-}f)$ produces terms proportional to $(2p^2 + m^2)/p$ that diverge at $p = 0$.
  This makes the second IBP inapplicable.
\end{enumerate}


%% ====================================================================
\section{Summary}
\label{sec:summary}
%% ====================================================================

The IBP improvement consists of one integration-by-parts step for each susceptibility $\chi_n$ ($n \geq 2$), using the identity
\begin{equation}
  \int_0^\infty p^2\,F_{k+1}(p)\,dp = T \int_0^\infty \frac{2p^2 + m^2}{E}\,F_k(p)\,dp\,,
\end{equation}
where $F_k = T\,\partial^k f/\partial\mu^k$.
Each step replaces an integrand with $k$ sign changes by one with $k{-}1$ sign changes, dramatically improving quadrature convergence at low~$T$.
The improvement is most pronounced for $\chi_3$ (from $\sim 100\%$ to $\sim 10^{-9}$ relative error) and significant for $\chi_4$ (from $\sim 400\%$ to $\sim 10^{-3}$ for interacting models).

For $\chi_2$, the IBP was already implemented earlier and replaces the $f(1{-}f)$ peak by the smooth Fermi step $f$, yielding excellent precision at all temperatures.

\end{document}
